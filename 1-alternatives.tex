\chapter{Is there any other way than using GPG ?}

\emph{NB : This article is much targeted to people with a special
interest in IT technology itself as an overview of secure communication
alternatives to GPG.\\Don't worry if you feel like this part is too
complicated or like it's going over your head. You can safely skip this
part or come there later when you understood more.}

So, here we are. We have OpenPGP and its software implementations, PGP
and GPG, which are tools to protect your mail from unwanted lookup. But
is it the best way ?

\subsection{Bitmessage}\label{bitmessage}

Let's take a look at \href{http://n0where.net/bitmessage/}{Bitmessage}
first. In my opinion, this is the perfect anti-example because it does
not respect the
\href{\{filename\}../Informatics/paradoxe-moquette-en.md}{carpet
paradox} : the addresses look like strings made out of random
characters, so they're a real hassle to exchange to your pen pals.

Even give your address on a piece of paper could be risky because your
friend may make a mistake while copying it. Like so easy to send your
message to someone else now !\\I think the best way to share a
Bitmessage address is to publish your address on a website you
control.\\However, even if you take this route, you still have to be
careful.

The foundation of Bitmessage itself is easy to understand, but as soon
as you dive deeper into it (a task I think is really important when
talking about safety protocols), you it gets confusing rather quickly.

\subsubsection{Some reflections on Bitmessage\ldots{}}\label{some-reflections-on-bitmessage}

One unusual thing about Bitmessage that comes to mind is the way it uses
P2P technology. The Bitmessage protocol works by sending a given message
to everybody in the network.

This makes it really less efficient with the energy. But it's also now
trivial for attackers like the NSA (the group hackers are typically
trying to escape from) to create their own Bitmessage address, collect
every message sent on the network and try to decrypt it because
apparently, it's
\href{http://www.nytimes.com/2013/09/06/us/nsa-foils-much-internet-encryption.html}{what
it already does} !

Instead of sending messages only to the recipient, the Bitmessage
protocol relly entirely on SSL/TLS cryptography to ensure privacy ! Not
the safest, as I will explain later.

\subsubsection{GPG or Bitmessage ?}\label{gpg-or-bitmessage}

\emph{This is my opinion, my use of computing tools. This is not some
sacred words to respect at all cost. A debate is possible. If you want
to troll or whatever, feel free, but without me !}

Given my above reflections, it's probably no surprise I prefer to use
GPG rather than Bitmessage. Mail is a widely recognized communications
channel, and even though I have ``no reason to hide'' - except privacy
concerns - I do understand that other people have motivations to want to
do so.\\This is especially true for those who want to hide who they are
writing to.

Use of Bitmessage marks me as a high-level \emph{hacker}, which I am
not. I can barely pretend to be a \emph{little} hacker or a
\emph{padawan}. Besides, using Bitmessage would further encourage people
to watch my email.

On the other hand, using signed or encrypted email, I still assume my
mail is being watched but I'm also able to protect it, along with my
privacy and the privacy of others. Since it is still plain email, many
more people can still use it for many more day-to-day purposes, without
learning whole new systems and while still being protected. Plus, the
more I use GPG, the more my use of it encourages other people to use
GPG, too.

And if an attacker wants to decrypt my mail, all they'd learn about me
is that I exchange thoughts with my lesbian friend, or some proposals
that I'm developing at work. Basically, useless stuff for NSA \& Co !

That said, there is a bonus: encrypting my mail is easy and takes no
extra effort on my part, but decrypting it does. Our Big Brother friends
have to spend enormous efforts and waste countless hours of CPU cycles
to decrypt things.

\begin{quote}
And make worse the life of others you don't like is really enjoyable !
\end{quote}

\subsection{Other classic email enciphering schemes}\label{other-classic-email-enciphering-schemes}

Another standard for encrypting and authenticating email is the S/Mime
standard, which uses SSL/TLS certificates very much like those used by
websites for encrypting Web traffic. Some entities and service providers
even offer free certificates for personal use.

It is the case of DanID, who mounted the authentication system NemID,
used by the danish governement and banks. It is valid to sign and
encrypt mail as well as authenticate on some websites, for example
\href{http://www.dba.dk/}{DBA}, the danish ebay, but you can't use it to
secure your website, sadly.

There's much to be said for a nation-wide system of secure certificates
like NemID, which are one step towards a digital citizen ID card, after
all. But the SSL/TLS protocol on which it is based have had numerous,
well-known problems lately. There's widespread concern that the NSA
(along with some other agencies) have been able to compromise the
certificate infrastructure or even break some of its crypto algorithms.

Personally, I tend not to trust TLS too much, because the more we look
at it, the more holes we see in it.

\textbf{\emph{Yet, I still recommand TLS use in web browsing ! A weak
shield is better than no shield at all !}}

SSL/TLS and GPG are a slightly different implementation of the same
principle : asymetrical cryptography, with a public and a private key.
But actually, I don't believe it to be easier and certainly not safer to
use SSL/TLS.

\subsection{Back to GPG}\label{back-to-gpg}

GPG tries to respect the carpet paradox: it uses well-known, easy to
understand protocols (the standard email protocol itself).\\The keys can
be identified by the mail address, or the fingerprint, easy to search
and giving a good feeling of security to users.

And, evidently, it's not been broken yet!