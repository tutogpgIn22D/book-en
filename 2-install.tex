\chapter{Softwares installation}

Let's do some actual stuff : install the softwares.

\section{What we need}\label{what-we-need}

To use GPG you need\ldots{}

\begin{enumerate}
\def\labelenumi{\arabic{enumi}.}
%\itemsep1pt\parskip0pt\parsep0pt
\item GPG
\item A keys manager like :

  \begin{itemize}
  \item Kgpg/Kleopatra
  \item Enigmail
  \end{itemize}
  
\item A mail software like :

  \begin{itemize}
  \item Kmail
  \item Thunderbird
  \item Evolution
  \end{itemize}
\end{enumerate}

Often, keys managers are add-ons or plugins to mail softwares. They have
pretty much the same options. Choosing one other the others is just a
matter of personal preferences.

Kgpg is a part of KDE, mostly used with Kontact and Kmail. Same,
Enigmail works with Thunderbird.

\begin{notice}
	The Samsungs devices with Android are now shipped with all the
	correct tools including GPG. You can use and create keys with it.\\The
	problem is that you can't really trust the hardware nor the software. I
	think it's a good idea to just begin. After that, you might want to set
	new keys using a true computer.
\end{notice}

By the way, you can also setup a personal mail server. It becomes
available to the non-computing guy with YunoHost\footnote{\url{https://yunohost.org/}} for example. Or in another way,
OpenBSD, really easy to install, yet a bit difficult to master. But, as I manage both systems now, I can tell it's not harder than a Debian!

\section{Recommendations}\label{recommendations}

\begin{quote}
Much of these recommendations are for general use, not only GPG.
\end{quote}

\subsection{Mail safety}

You ought to avoid webmail. Webmail is a bad thing, as you access your mail through the web, so we can't ensure of safety, without even
thinking of GPG itself!

Encrypting and signing mail does not protect you from viruses ! They
warrant that your mail is authentic and/or has not been read by anyone
but the recipient.

\subsection{Softwares safety}

When you can use free or open-source software, please do so. It's even more important in critical protocols like security ones. Being open, the code can be read by anyone and anybody can tell that there's no backdoors or bad habits. An \emph{opensource} code means that you can trust it to \emph{protect yourself, and your privacy with it} ! And if you're really a paranoiac (your right) then you can take code cursus and then do this checking yourself!

When installing or setting a software, don't hit ``enter'' all the time. Check the options. It is frequent that an install program contains a
toolbar or something else you did not ask nor need. These micro add-ons are one of the main source of your computer being slow, buggy, and can contain viruses or spy softwares. You also ought to avoid fetching softwares on third-parties websites but rather on the author's one, or an official website like Apple's one in case of you being a MacOS user.

Some software writers sign their binaries (the file to download) with their GPG key or indicate MD5 or SHA1 sums. I have not told you yet how to check these signatures, but if you know how to do it, please do!

\section{GPG}\label{gpg}

\begin{notice}
As said somewhere else, \emph{OpenPGP} is the standard, the rule describing the whole protocol (messages formats, ciphers to use, signing or encrypting operations…). \emph{GPG}, or \emph{GnuPG} is a software designed to comply with this standard. It is this software that will perform all technical operations.\\
\\
Later, I can say \emph{OpenPGP} or \emph{GPG key}. It is the same.\\
\\
When talking about the general operation, meaning the protocol, I will say \emph{OpenPGP}. When saying \emph{GPG}, I refer to the software itself.
\end{notice}

\subsection{Linux and co}\label{linux-and-co}

Gpg is somewhat a standard in any GNU/Linux distribution. If you don't have it already, it means your distro is that weird I even don't know
much about it, nor its packaging tool ( \emph{Slitaz} ? )

Under Debian \& co, if not already installed - which would be really weird, as Debian uses gpg to sign and check its softwares packages (!) - it is it:

\begin{lstlisting}
apt-get install gnupg gnupg2
\end{lstlisting}

The second release is the recommended one currently.

No matter, it's in the general dependencies of all key managers we'll
see later. Users of other distros or graphical tools like Synaptic, Apper or Muom will simply make a research about \textbf{gnupg}. Much chance is that
your package manager tell you it's already there.

It's also in BSD base, like on OpenBSD.

\subsection{Windows}\label{windows}

This is a big one !

You need Gpg4win\footnote{\url{http://www.gpg4win.org/download.html} - take the
one on the top, unless you know what you're doing}, which
actually contains all the needed softwares\footnote{\url{http://www.gpg4win.org/about.html}} : keys manager, mail client\ldots{}

Windows users, you have an easy life ! As soon as the installer is downloaded, launch it. It will propose you to install GPG and other softwares.

\begin{itemize}
\itemsep1pt\parskip0pt\parsep0pt
\item
  GPA is a keys manager.
\item
  Kleopatra is another keys manager.
\end{itemize}

You need one of them. Kleopatra is the most documented on thee web, so I
recommend it. I use it from time to time and if you need help, it will
be easier for me if you use this one.

\begin{itemize}
\item GpgOL, Outlook plugin
\item GpgEX, Windows files explorer plugin.
\end{itemize}

Install them if you want or need.

\begin{itemize}
\item Claws-Mail, lightweight mail client
\end{itemize}

\subsection{MacOS}\label{macos}

Thunderbird is available on Mac and the native mail client can also support encryption.\\

You need the Gpg Suite\footnote{\url{https://gpgtools.org/}}. Please download it. You can (actually you \emph{should}) check dmg file integrity by going
in your Downloads repository (I suppose here it's called
\emph{Downloads}) with your Terminal:

\begin{lstlisting}
cd Downloads
openssl sha1 GPG_Suite\ldots{}
\end{lstlisting}

Typing filename, you can use auto-completion : use tab key, the terminal will complete the filename itself.

The ssl command will generate a string of characters which have to correspond to the one indicated on the website, just under download
button. If it fails, bare download again.

So you can install now. Open the dmg file and select the correct options. You need \emph{MacGPG2}, \emph{GPGPreferences}, \emph{GPG Keychain Access}.

If you use \textbf{Mail}, the native client, you need \textbf{GPG for Mail} but if you use Thunderbird, you need \textbf{GPG Services}.

\section{A mail client}\label{a-mail-client}

\begin{notice}
I won't describe mail and address configuration. If you came there, I
believe you're motivated to learn/search the solution yourself or you
already know how to do it.

Yet, one can still contact me\footnote{through the tutorial address or \url{http://www.22decembre.eu/en/contact.html}} to
ask for help and a basic tutorial for setting Thunderbird is available on its website\footnote{\url{https://support.mozilla.org/en-US/kb/new-email-address}}.
\end{notice}

So, you have the base software, but nothing else currently. Choose your mail client.

\subsection{Thunderbird}\label{thunderbird}

Thunderbird is available for downloading for most languages and platforms.\footnote{\url{https://www.mozilla.org/en-US/thunderbird/} (link for your language and your operating system)}

So you can install it with apt\footnote{It has been renamed Icedove in Debian due to the Mozilla-Debian issue : 
	\url{http://en.wikipedia.org/wiki/Mozilla_Corporation_software_rebranded_by_the_Debian_project}.}:

\begin{lstlisting}
apt-get install icedove
\end{lstlisting}

\subsection{Kmail and Evolution}\label{kmail-and-evolution}

Kmail is a part of KDE, Evolution is a part of Gnome. So if you are
under GNU/Linux, you should use your favorite package manager.

\begin{lstlisting}
apt-get install kmail

apt-get install evolution
\end{lstlisting}

Same as before : Synaptic or other graphical installers will setup all
dependencies, including \textbf{gnupg} if for some weird reason it's not
already the case.

\subsection{Claws-Mail}\label{claws-mail}

If you are on Windows, remember that the Gpg4win installer proposed you
to install as well \textbf{Claws-Mail}.

\subsection{The others}\label{the-others}

There is a Windows version of KDE\footnote{\url{https://windows.kde.org/}}, but I never tried to use it - I feel already really fine on GNU/Linux and
don't see why I should change that. Sylpheed, mail client available under Linux distributions, Windows, Mac
and other Unix.

\section{A keys manager}\label{a-keys-manager}

\begin{notice}
To remind you : You need only one of these softwares. Often the choice
much depends on your environment (operating system).
\end{notice}

\subsection{GPG Keychain under Mac OS}\label{gpg-keychain-under-mac-os}

The keys manager is named GPG Keychain under Mac Os and you have already
installed it normally when you installed Gpg for Mac.

\subsection{Thunderbird : Enigmail}\label{thunderbird-enigmail}

If you use Thunderbird, you will need Enigmail, which is actually a
plugin used by Thunderbird to interact with Gpg.

\subsubsection{First way : Downloading on the website}\label{first-way-downloading-on-the-website}

Installation here is the same as a Firefox module : xpi. You need to
download the module\footnote{\url{https://www.enigmail.net/download/}} and
of course take the one corresponding to your operating system.

The xpi extension setup as described there : you need to launch
Thunderbird, select «Tools» in the menus bar on top, then
«add-ons»,«plugins», «extensions» or «modules». Or, depending on your
version of Thunderbird, select the big button on top and right and
select «add-ons».

Here, you can tell Thunderbird that you want to install a plugin
clicking on the bottom left on «install\ldots{}». Thunderbird will ask
you where you downloaded the xpi.

\subsubsection{Second way : Install with Thunderbird itself}\label{second-way-install-with-thunderbird-itself}

You can ask Thunderbird to install it itself. Like previously, go to the add-ons window and make a research about
Enigmail. The software should appears on top of the list with an install
button.

Once you installed the plugin, you need to reboot Thunderbird itself.
That's it.

\subsubsection{Debian Bonus : installation via Apt}\label{debian-bonus-installation-via-apt}

If you use Icedove (sic !) you can also install Enigmail with apt. It is
good if you share your computer with other users. But if you do it this way, take care when upgrading : don't upgrade on
Mozilla request but bare wait for Debian to update it! If you wish to use another release than the one in Debian apt store,
uninstall it using apt before.

\begin{warning}
One only release of this software per machine ! I believe
it to be a safety mesure.
\end{warning}

\begin{lstlisting}
apt-get install enigmail
\end{lstlisting}

\begin{notice}
If Debian did it, why not others like Mageïa or Arch ? Try to
find Enigmail in your packages manager.
\end{notice}

\subsection{Kgpg/Kleopatra}\label{kgpgkleopatra}

Kgpg and Kleopatra are two softwares used to manage keys and
certificates in KDE. I prefer to use Kgpg, but it's still useful to have
both installed. If you are a KDE fan like I am, then chances are that they are already
there. Else, as previously :

\begin{lstlisting}
apt-get install kgpg kleopatra
\end{lstlisting}

Remember, if you are under Windows, you got Kleopatra in Gpg4win
installer.

\subsection{The others}\label{the-others-1}

Under Gnome, you have Seahorse, which I don't know at all. I doubt that
it is really different from Kgpg. Available under Debian :

\begin{lstlisting}
apt-get install seahorse
\end{lstlisting}