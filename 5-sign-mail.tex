\chapter{Sign your mail}

So you followed the path and got a gpg key. Wonderful. Now, you will learn how to use your key with your mail.

\section{How to make your various pen pals use your key ?}\label{how-to-make-your-various-pen-pals-use-your-key}

\subsection{Config' of mail client}\label{config-of-mail-client}

First of all, you have to tell your various pen pals about GPG. A little text on the
bottom of your mail will be enough. This text is called
\emph{signature}. Not to be confused with your GPG signature on your
mail itself.

Here is my English signature for example:

\begin{quoting}
The file \textbf{\emph{signature.asc}} is not attached to be read by
you. It's a digital signature by GPG.\\If you want to know why I use it,
and why you should as well, you can read my article
there:\\http://www.22decembre.eu/2015/03/21/introduction-en/
\end{quoting}

This text is to be filled in your mail client config'. You will set
there also your various choices for your GPG signatures.

\subsubsection{Kmail}\label{kmail}
It's \textbf{\emph{Settings \textgreater{} Configure
Kmail\ldots{} \textgreater{} Identities}}. There, you will find cryptography options, where you set which
\textbf{\emph{private}} key to use to sign your mails. You will also indicate which \textbf{\emph{public}} key is to use when
encrypting mail to yourself (pretty good way to share information across
devices, like a wifi password !) About format, it is better to use
\emph{OpenPGP/Mime} rather than \emph{inline}. It is also important to set your preference about mail composing in
\textbf{\emph{Settings \textgreater{} Configure Kmail\ldots{}
\textgreater{} Composing}}. I checked almost everything except \emph{Always show the encryption keys
for approval}. The options are pretty well described for you to
understand if you want to use them or not.

\subsubsection{Thunderbird}\label{thunderbird}
Thunderbird has a documentation support on its website\footnote{\url{https://support.mozilla.org/en-US/kb/digitally-signing-and-encrypting-messages}}.

In \textbf{\emph{Tools \textgreater{} Account Settings}}, select
\emph{OpenPGP Security} under the address you want. Select the option
\emph{Enable OpenPGP support (Enigmail) for this identity} and then
\emph{Use email address of this identity to identify OpenPGP key}.

You can then choose your defaults settings : encrypt, sign or not, and
if you want to use \emph{PGP/Mime}, which I recommend. You have also some good options to set in \textbf{\emph{Enigmail \textgreater{} Preferences}}.

\subsubsection{Evolution}\label{evolution}
\emph{Edit \textgreater{} Preferences}

Select \emph{Mail accounts}, then the needed account and click
\emph{Edit}. In the following account editor, go to the \emph{Security}
tab on the far right, and in the field \emph{PGP/GPG Key ID} : copy the
8 characters ID that your keys manager gave you. Remember to set your
default options. When writing a new mail, in the \emph{Security} menu, click on \emph{PGP
Sign} and/or \emph{PGP encrypt}.

\section{Shall you sign and encrypt all your mail ?}\label{shall-you-sign-and-encrypt-all-your-mail}

This question is more of ethical nature. It is a personal choice. Anyway, your software has certainly some big buttons that just want to
be used to realise the cryptographic operations.

I like Phil Zimmerman's thoughts on the matter:

\begin{quoting}
What if everyone believed that law-abiding citizens should use postcards
for their mail? If a nonconformist tried to assert his privacy by using
an envelope for his mail, it would draw suspicion. Perhaps the
authorities would open his mail to see what he's hiding. Fortunately, we
don't live in that kind of world, because everyone protects most of
their mail with envelopes. So no one draws suspicion by asserting their
privacy with an envelope. There's safety in numbers. Analogously, it
would be nice if everyone routinely used encryption for all their email,
innocent or not, so that no one drew suspicion by asserting their email
privacy with encryption. Think of it as a form of solidarity.
\end{quoting}

Remember, you can always sign all your mail. It's pretty rare that it
will cause trouble to your pen pals (mostly because they have a bad/old
email client).

A signed mail is authentic. Your pen pal will be sure that it
comes from you and that it has not been changed on the path. But it's
still possible for everybody to read it.


But you can encrypt your mails only if your pen pals use GPG also,
because you need their public key to encrypt mail for them.

\section{Exercise}\label{exercise}

I am going to ask you to simply send me a signed mail. Bare
simple ! This is why I asked you to send me your key in the last article: I need
it to check your signature. Checking your signature, I can insure you
that you understood this part of the tutorial. So, let's do it : write a mail to \emph{Tuto-gpg @ 22decembre.eu}. You
can say what ever you would like to say, send a picture, make a comment
about the tutorial, say that you love me\ldots{}

Before sending, use the \emph{Sign} option, or the button. If, like me,
you setted your mail software to sign all your mail, you have nothing
more to do than hitting the \emph{send} button now. I am waiting forward reading from you soon.
