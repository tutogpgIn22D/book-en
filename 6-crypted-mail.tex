\chapter{Read and write encrypted mail}

Let's go further. You will encrypt your mail. It was the first objective of this article series. Yet, there is more to come !
Did you followed the previous steps ? So, I must have sent you a mail. It is encrypted and you are the only one able to read it.

\section{What did I do ?}\label{what-did-i-do}

When I received your public key, I imported it in my keyring.\\It
allowed me to analyze it and mail you back this answer: \emph{yes, your
key is fine}.

\section{Consequences}\label{consequences}

As I received your public key, I can now send you encrypted mail.

\textbf{If you received an encrypted mail, you are normally the only person able to read it. But it does not tell you about the sender.}

But if I sign it, you can't check the signature, as you need my public key to do so, and I have not yet told you how to do it.

\textbf{Signing a mail means that I am the true sender of it.}

This is the reason your mail software says that this mail is signed with
gpg, but it can't check the signature.

\section{Some little caution}\label{some-little-caution}

Be cautious. Only the main body of the mail is encrypted. The header, subject and other meta-datas (who sent the mail, to whom\ldots{}) are
clair and readable text. It is hard or even impossible to do in another way: how mail servers could know whose mail it is and who to give it if
destination is not available ?

Thus, it is good practice to set a neuter subject if you want to insure good confidentiality.

Something like «a good plan» rather than «the great plan to world's domination !» or «last production figures» rather than «prod' figures
rising by 50 \%: everything is good baby !»

\section{How to get the public key ?}\label{how-to-get-the-public-key}

\subsection{By mail}\label{by-mail}

I could have send you the public key by mail in the same way as you did. But this is not safe: a malicious person could have hijacked your mail and replaced your key by another. This is the kind of thoughts I would like you to have: safety on internet is a process, a way of thinking.

\subsection{On the web}\label{on-the-web}

You can place your key on a webpage if you have one. In that case, it is good policy to indicate the fingerprint. A key fingerprint is a series of numbers and characters unique to the key. It allows to identify (quite safely) a key and same time insure it of its integrity. Your keys manager can give you this fingerprint. It looks like this :

\begin{center}
	\oldstylenums{30CF 1DA5 7E87 6BAA 730D E561 42E0 A02E F1C9 35A4}
\end{center}

This fingerprint is the one of \emph{Tuto-gpg @ 22decembre.eu} GPG key. It is the same principle as MD5 sums\footnote{\url{http://en.wikipedia.org/wiki/MD5}} for downloaded files other the net - Linux distribution iso image for example. Some persons also set their key fingerprint on their business card, easier to give them.

\subsection{On the keys servers}\label{on-the-keys-servers}

Knowing a key fingerprint, it is also possible to find it on the
internet ! Actually, on such servers called «keys servers» or «gpg
servers». Here are some servers:

\begin{itemize}
\itemsep1pt\parskip0pt\parsep0pt
\item
  hkp://keyserver.ubuntu.com/
\item
  hkp://pool.sks-servers.net/ NB : you will certainly visit their
  \href{https://sks-keyservers.net/i/}{website}.
\item
  hkp://pgp.mit.edu/
\item
  hkp(s)://keys.gnupg.net/
\end{itemize}

You can tell gpg which server to use first with your keys manager settings. The S indicate that you use the server with a secure TLS connexion. This way, if you're a paranoiac, no one knows which keys you are searching. These servers allow you to make your key public, find keys of people you don't know, and finally check their mails signatures.

\section{Exercise}\label{exercise}

The exercise is to find the tutorial key and send me an encrypted mail.

\subsection{Find the key}\label{find-the-key}

As you understood it, the aim is to make you understand the way keys servers work.

Open your keys manager, and then the dialog box to servers. It will ask you for a string to search, meaning a fingerprint or a mail address.
You can copy-paste the key fingerprint or the mail address (both written above). Your keys manager will then show you some keys to download.
Either you wrote the mail address, and shall check the fingerprint, or opposite, copied the fingerprint and shall check the address. I actually
created several keys, not to screw you but to make you think and work your logic.

This tutorial aims at teach you to use GPG. So you have to use it and confront it.

By the way, please note that these persons signed the key:

\begin{itemize}
\itemsep1pt\parskip0pt\parsep0pt
\item
  \href{http://alterlibriste.free.fr/}{alterlibriste}
\item
  \href{http://lehollandaisvolant.net/}{le hollandais volant}
\end{itemize}

They helped me in some ways, because I could use their texts, or they encouraged me to write this tutorial. These other persons also helped me:

\begin{itemize}
\itemsep1pt\parskip0pt\parsep0pt
\item
  \href{http://genma.free.fr/}{genma}
\item
  \href{https://maymay.net/}{Maymay} who helped with some of the English
  texts.
\end{itemize}

I wish to thank them all.

Now, look at the mail I send you: your mail software might have change its announcement!

\subsection{Write an encrypted mail}\label{write-an-encrypted-mail}

You have to write me an encrypted mail. You can also sign it, but it is not the aim of this part. Do as you want.
Just before sending it, select the \emph{Encrypt} option or button. Here it is !
Maybe your mail software even proposed you to encrypt the mail while writing it because you now have the needed key.
Isn't it beautiful?